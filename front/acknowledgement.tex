% !TeX root = ../main.tex

\begin{acknowledgement}

常到外國朋友家吃飯。當蠟燭燃起,菜肴布好,客主就位,總是主人家的小男孩或小女孩舉起小手,低頭感謝上天的賜予,並歡迎客人的到來。

我剛到美國時,常鬧得尷尬。因為在國內養成的習慣,還沒有坐好,就開動了。

以後凡到朋友家吃飯時,總是先囑咐自己;今天不要忘了,可別太快開動啊!幾年來,我已變得很習慣了。但我一直認為只是一種不同的風俗儀式,在我這方面看來,忘或不忘,也沒有太大的關係。

前年有一次,我又是到一家去吃飯。而這次卻是由主人家的祖母謝飯。她雪白的頭髮,顫抖的聲音,在搖曳的燭光下,使我想起兒時的祖母。那天晚上,我忽然覺得我平靜如水的情感翻起滔天巨浪來。

在小時候,每當冬夜,我們一大家人圍著個大圓桌吃飯。我總是坐在祖母身旁。祖母總是摸著我的頭說:「老天爺賞我們家飽飯吃,記住,飯碗裡一粒米都不許剩,要是蹧蹋糧食,老天爺就不給咱們飯了。」

剛上小學的我,正在念打倒偶像及破除迷信等為內容的課文,我的學校就是從前的關帝廟,我的書桌就是供桌,我曾給周倉畫上眼鏡,給關平戴上鬍子,祖母的話,老天爺也者,我覺得是既多餘,又落伍的。

不過,我卻很尊敬我的祖父母,因為這飯確實是他們掙的,這家確實是他們立的。我感謝面前的祖父母,不必感謝渺茫的老天爺。

這種想法並未因為年紀長大而有任何改變。多少年,就在這種哲學中過去了。

我在這個外國家庭晚飯後,由於這位外國老太太,我想起我的兒時,由於我的兒時,我想起一串很奇怪的現象。

祖父每年在「風裡雨裡的咬牙」,祖母每年在「茶裡飯裡的自苦」,他們明明知道要滴下眉毛上的汗珠,才能撿起田中的麥穗,而為什麼要謝天?我明明是個小孩子,混吃混玩,而我為什麼卻不感謝老天爺?

這種奇怪的心理狀態,一直是我心中的一個謎。

一直到前年,我在普林斯頓,瀏覽愛因斯坦的我所看見的世界得到了新的領悟。

這是一本非科學性的文集,專載些愛因斯坦在紀念會上啦,在歡迎會上啦,在朋友的喪禮中,他所發表的談話。

我在讀這本書時忽然發現愛因斯坦想盡量給聽眾一個印象:即他的貢獻不是源於甲,就是由於乙,而與愛因斯坦本人不太相干似的。

就連那篇亙古以來嶄新獨創的狹義相對論,並無參考可引,卻在最後天外飛來一筆,「感謝同事朋友貝索的時相討論。」

其他的文章,比如奮鬥苦思了十幾年的廣義相對論,數學部份推給了昔年好友的合作:這種謙抑,這種不居功,科學史中是少見的。

我就想,如此大功而竟不居,為什麼?像愛因斯坦之於相對論,像我祖母之於我家。

幾年來自己的奔波,做了一些研究,寫了幾篇學術文章,真正做了一些小貢獻以後,才有了一種新的覺悟:即是無論什麼事,得之於人者太多,出之於己者太少。因為需要感謝的人太多了,就感謝天罷。無論什麼事,不是需要先人的遺愛與遺產,即是需要眾人的支持與合作,還要等候機會的到來。越是真正做過一點事,越是感覺自己的貢獻之渺小。

於是,創業的人,都會自然而然的想到上天,而敗家的人卻無時不想到自己。

\end{acknowledgement}