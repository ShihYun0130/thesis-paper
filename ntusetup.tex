% !TeX root = ./main.tex

% --------------------------------------------------
% 資訊設定(Information Configs)
% --------------------------------------------------

\ntusetup{
  university*   = {National Taiwan University},
  university    = {國立臺灣大學},
  college       = {管理學院},
  college*      = {College of Management},
  institute     = {資訊管理學研究所},
  institute*    = {Department of Information Management},
  title         = {預測社群媒體中品牌官方發佈內容的人氣},
  title*        = {Popularity Prediction of Online Content from Official Brands Channels on Social Media},
  author        = {陳詩筠},
  author*       = {Shih-Yun Chen},
  ID            = {R09725037},
  advisor       = {魏志平},
  advisor*      = {Chih-Ping Wei},
  date          = {2022-07-01},         % 若註解掉,則預設為當天
  oral-date     = {2022-07-01},         % 若註解掉,則預設為當天
  DOI           = {10.5566/NTU2022XXXXX},
  keywords      = {LaTeX, 中文, 論文, 模板},
  keywords*     = {LaTeX, CJK, Thesis, Template},
}

% --------------------------------------------------
% 加載套件(Include Packages)
% --------------------------------------------------

\usepackage[sort&compress]{natbib}      % 參考文獻
\usepackage{amsmath, amsthm, amssymb}   % 數學環境
\usepackage{ulem, CJKulem}              % 下劃線、雙下劃線與波浪紋效果
\usepackage{booktabs}                   % 改善表格設置
\usepackage{multirow}                   % 合併儲存格
\usepackage{diagbox}                    % 插入表格反斜線
\usepackage{array}                      % 調整表格高度
\usepackage{longtable}                  % 支援跨頁長表格
\usepackage{paralist}                   % 列表環境
\usepackage[toc,page]{appendix}


\usepackage{lipsum}                     % 英文亂字
\usepackage{zhlipsum}                   % 中文亂字

% --------------------------------------------------
% 套件設定(Packages Settings)
% --------------------------------------------------
