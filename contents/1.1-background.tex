% !TeX root = ../main.tex
% \graphicspath{ {./images/} }
\section{Background}

With over 1.5 billion users worldwide, YouTube is currently one of the largest video-sharing platforms, where more than 500 hours of videos are uploaded per hour. It is not hard to imagine that launching videos to YouTube to promote brands and create trending topics worldwide is a common marketing strategy for businesses. YouTube is also regarded as a social media where users can create their channels and interact with other users with a "user-content-user" through a gluing layer of uploaded content opposing to traditional "user-user" interaction. Consequently, it is also reasonable for both academic and industrial to have put significant efforts into researching popularity prediction and patterns of information diffusion on YouTube.

Information diffusion describes how information is spread from one node to another in a network. Researches and observations of information diffusion are often conducted in various application domains. Examples of real-world applications include the performance of a marketing campaign or advertisement, the estimation of the dynamic popularity of online content, the prevention of an epidemic disease, the measuring of the spreading of news or rumor, the prospective impact of an idea, etc. Cascade is the basis of information diffusion. By structuring the information cascade of the information diffusion in a network, we can understand the crucial factors of a rapidly spreading information in a network and thus have the ability to predict the total population that a piece of propagation information can affect at any specific time.

The vigorous development of social media has changed the way people interact with brands. Users get information about brands through the posts or videos released to social media; further, they can express their feelings by leaving comments or simply clicking the like and dislike button under posts. As to businesses, they share content and launch marketing campaigns on social media in order to increase their brand awareness, create a positive brand association, and improve communications and interactions with their potential customers. Moreover, understanding how a brand campaign will perform can help businesses decide which marketing plan they are going to execute. For example, comparing a video having a short lifespan to a longer one, it may be suitable for some brands to choose the former one. That is to say, to release only one video that is expected to generate trending topics lasting for a long time. Therefore, it is valuable to explore the evolution of information popularity by structuring and understanding the information diffusion of online content in a social network.

% ref: The YouTube Social Network (2012)


