% !TeX root = ../main.tex
% \graphicspath{ {./images/} }
\section{Motivation and Objective}

Research on popularity prediction has attracted attention in the past decades due to its essential involvement in daily life. Various methodologies are prosperously developed to predict the size of the population by structuring the cascade of information diffusion. The majority of existing research has proposed feature-based models using feature engineering approaches to extract critical factors to predict the popularity of information cascade. 

While some feature-based models are proved to efficiently explain whether features such as temporal features, user attributes, and content have significant effects on deciding the popularity, they are sometimes insufficient to figure out every possibility between features and popularity. More specifically, by simply building feature-based models, we may have ignored some important details and overlooked the relationship between factors and the outcome of information diffusion. 

Hence, we propose a model with both deep learning-based approaches and graph representation-based approaches to better identify the critical factors to the popularity evolution of videos published to YouTube official channels of brands. The objective of this paper include:

\begin{itemize}
\setlength{\itemsep}{0pt}
\setlength{\parskip}{0pt}
\setlength{\topsep}{0pt}
\setlength{\partopsep}{0pt}
    \item Predicting the overall popularity size that a brand campaign can affect in a social network at any given time by capturing the influential factors of an official published video.
    \item Accurately using graph-based approaches to model time series trends and user relationships in a complicated network.
\end{itemize}

The rest of this thesis is organized as follows:

\begin{itemize}
\setlength{\itemsep}{0pt}
\setlength{\parskip}{0pt}
\setlength{\topsep}{0pt}
\setlength{\partopsep}{0pt}
    \item We review related works of various approaches of modeling information diffusion on different social networks in Chapter \ref{chap:2}.
    \item Chapter \ref{chap:3} introduces our dataset and the proposed model structuring the dynamic popularity in a social network.
    \item The experiment details and results will be presented and summarized in chapter \ref{chap:4}.
    \item In the last section, we will discuss the contribution and the limitation in this thesis and share some insight and future work in chapter \ref{chap:5}.
\end{itemize}
